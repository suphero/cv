\resumeSubheading
  {Backend Takım Lideri · Superapp Backend}
  {Ozan}
  {Uzaktan}
  {Ağustos 2023 -- Günümüz}
\begin{itemize}
  \item Uzak backend ekibini yöneterek sprint planlama, backlog önceliklendirme ve fintek ürünleri için teslimat süreçlerini yönlendiriyorum.
  \item Java, Spring Boot ve PostgreSQL kullanarak regülasyon gereksinimlerini karşılayan ölçeklenebilir mikro servis mimarileri tasarlıyorum.
  \item Çapraz fonksiyonel ekiplerdeki engelleri kaldırıyor ve çevik süreçleri iyileştirerek teslimat sürelerini kısaltıyorum.
\end{itemize}
\vspace{0.3em}

\resumeSubheading
  {Kıdemli Yazılım Geliştirici · Superapp Backoffice}
  {Ozan}
  {Uzaktan}
  {Aralık 2022 -- Ağustos 2023}
\begin{itemize}
  \item Nest.js, TypeScript ve GraphQL ile Backend for Frontend katmanını geliştirerek entegrasyonları tek noktadan yönettim.
  \item Geliştirici dostu mimari desenleri ve dokümantasyonu yaygınlaştırarak hata ayıklama ve işe alıştırma süresini azalttım.
  \item Veri akışlarını optimize ederek gecikmeleri minimize ettim ve arka ofis kullanıcı deneyimini iyileştirdim.
\end{itemize}
\vspace{0.3em}

\resumeSubheading
  {Yazılım Mimarı · Mobil Arama ve Chatbot}
  {Akbank}
  {Uzaktan}
  {Kasım 2020 -- Aralık 2022}
\begin{itemize}
  \item Microsoft Bot Framework kullanarak Akbank chatbot ürününün arka uç mimarisini tasarlayıp geliştirdim.
  \item ETL hatlarını ve arama indekslerini optimize ederek mobil aramada doğruluk ve yanıt hızını artırdım.
  \item Mülakat süreçlerine katılarak ekibin teknik yetkinliğini güçlendirecek aday seçimlerinde rol aldım.
\end{itemize}
\vspace{0.3em}

\resumeSubheading
  {Kıdemli Yazılım Uzmanı · D-Filo}
  {Doğuş Teknoloji}
  {İstanbul, Türkiye}
  {Eylül 2020 -- Kasım 2020}
\begin{itemize}
  \item Yeni filo yönetimi çözümünün altyapı ve arka uç platformunu sıfırdan birlikte tasarladım.
\end{itemize}
\vspace{0.3em}

\resumeSubheading
  {Kıdemli Yazılım Uzmanı · VDF Filo}
  {Doğuş Teknoloji}
  {İstanbul, Türkiye}
  {Nisan 2016 -- Eylül 2020 (2018 hariç)}
\begin{itemize}
  \item .NET Core ve TypeScript ile faturalama, operasyon ve tahsilat modüllerini içeren filo yönetimi çözümleri geliştirdim.
  \item Projenin ikinci fazında CI/CD süreçlerini olgunlaştırarak dağıtımları otomatikleştirdim.
  \item Operasyonel performansı artıran yeni işlevlerin sürekliliğini sağladım.
\end{itemize}
\vspace{0.3em}

\resumeSubheading
  {Yazılım Uzmanı · NTV \& NTV Spor}
  {Doğuş Teknoloji}
  {İstanbul, Türkiye}
  {Eylül 2018 -- Aralık 2018}
\begin{itemize}
  \item NTV ve NTV Spor uygulamaları için modern teknolojileri ve NoSQL veri depolarını deneyerek çözümler geliştirdim.
\end{itemize}
\vspace{0.3em}

\resumeSubheading
  {Yazılım Uzmanı · Doğuş İnşaat ERP}
  {Doğuş Teknoloji}
  {İstanbul, Türkiye}
  {Ocak 2018 -- Eylül 2018}
\begin{itemize}
  \item ERP arayüzlerini yeniden tasarlayarak veri giriş süreçlerini sadeleştirdim ve raporlama iş akışlarını hızlandırdım.
\end{itemize}
\vspace{0.3em}

\resumeSubheading
  {Yazılım Geliştirici · Leaseplan ERP}
  {Doğuş Teknoloji}
  {İstanbul, Türkiye}
  {Temmuz 2013 -- Nisan 2016}
\begin{itemize}
  \item .NET Framework, MSSQL ve JavaScript kullanarak araç kiralama ERP platformu için uçtan uca özellikler geliştirdim.
\end{itemize}
