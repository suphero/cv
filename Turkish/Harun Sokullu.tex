%%%%%%%%%%%%%%%%%%%%%%%%%%%%%%%%%%%%%%%%%
% Important notes:
% This template needs to be compiled with XeLaTeX and the bibliography, if used,
% needs to be compiled with biber rather than bibtex.
%%%%%%%%%%%%%%%%%%%%%%%%%%%%%%%%%%%%%%%%%

\documentclass[]{../friggeri-cv} % Add 'print' as an option into the square bracket to remove colors from this template for printing
\usepackage{fontawesome}

\begin{document}

\header{harun}{sokullu}{yazılım uzmanı}

\begin{aside}
\section{iletişim}
Çağ 11/11
Çeliktepe, 34413
Kağıthane / İstanbul
Türkiye
~
+90 (507) 471 4044
~
\href{mailto:harunsokullu@gmail.com}{harunsokullu@gmail.com}
\href{https://www.linkedin.com/in/suphero}{\faLinkedin/suphero}
\href{https://github.com/suphero/cv/raw/master/Turkish/Harun\%20Sokullu.pdf}{\faGithub/suphero/cv}
\section{proje linkleri}
\href{https://github.com/suphero}{\faGithub/suphero}
\href{https://play.google.com/store/apps/developer?id=Harun+Sokullu}{\faAndroid/Harun Sokullu}
\href{https://itunes.apple.com/tr/developer/harun-sokullu/id1265151811}{\faApple/Harun Sokullu}
\section{programlama dilleri}
C\#, T-SQL, TypeScript, JavaScript, \LaTeX, Python
\section{kütüphane}
Entity Framework, ASP.NET, AngularJS, ionic, OpenCV, TensorFlow, Firebase
\section{ide}
Visual Studio, SSMS, VSCode, Atom, Unity, Google Apps Script
\section{diller}
Türkçe, İngilizce
\end{aside}

\section{deneyim}

\begin{entrylist}

\entry
{2015--Şimdi}
{Doğuş Teknoloji}
{}
{Yazılım Uzmanı}

\entry
{2013--2015}
{Doğuş Teknoloji}
{}
{Yazılım Mühendisi}

\end{entrylist}

\section{kurumsal projeler}

\begin{entrylist}

\entry
{2018--Şimdi}
{Agrega}
{Doğuş Teknoloji}
{Doğuş İnşaat ERP Uygulaması
\\C\#, .NET Core, MVC, T-SQL, WCF, Git}

\entry
{2016--2018}
{FleetUp}
{Doğuş Teknoloji}
{VDF Filo ERP Uygulaması
\\C\#, TFS, MVC, TypeScript, AngularJS, T-SQL, WCF, Entity Framework, LINQ}

\entry
{2013--2016}
{FleetNet}
{Doğuş Teknoloji}
{LeasePlan ERP Uygulaması
\\C\#, TFS, ASP.Net, T-SQL, LINQ}

\end{entrylist}

\section{kişisel projeler}

\begin{entrylist}

\entry
{2017-2018}
{Borsa Tahminleyici}
{Python, TensorFlow, Makine Öğrenmesi, Elasticsearch}
{Makine Öğrenmesi Algoritmalarıyla Borsa Tahminleme}

\entry
{2017-2018}
{CV}
{\LaTeX}
{\href{https://github.com/suphero/cv}{\faGithub/cv}}

\entry
{2017}
{Google Forms to SQL}
{Google Forms, Google Scripts}
{Google Forms kullanarak SQL sorgusu üretici
\\\href{https://github.com/suphero/Google-Forms-to-SQL}{\faGithub/Google Forms to SQL}}

\entry
{2017}
{CugiC}
{Unity}
{NOT NOT oyun replikası}

\entry
{2017}
{Hotel OCR}
{OpenCV, Python}
{OpenCV tabanlı kimlik okuyucu}

\entry
{2017}
{Hız Koridoru}
{ionic, Firebase}
{Otoyol Hız Koridoru sorgulama ve gişe geçiş süresi asistanı
\\\href{https://play.google.com/store/apps/details?id=com.harunsokullu.speedcorridor}{\faAndroid/Hız Koridoru}
\\\href{https://itunes.apple.com/tr/app/h\%C4\%B1z-koridoru/id1265151812}{\faApple/Hız Koridoru}}

\entry
{2016--2017}
{CBDownloader}
{C\#}
{Canlı Yayın Video indirme uygulaması
\\\href{https://github.com/suphero/CBDownloader}{\faGithub/CBDownloader}}

\entry
{2011}
{Tarifini Bul}
{Android}
{Yemek tarifi bulucu mobil uygulama
\\\href{https://play.google.com/store/apps/details?id=com.tarifinibul}{\faAndroid/Tarifini Bul}}

\end{entrylist}

\section{eğitim}

\begin{entrylist}

\entry
{2018--Şimdi}
{Bilgisayar Mühendisliği}
{Yüksek Lisans}
{Ahmet Yesevi Üniversitesi}

\entry
{2006--2011}
{Telekomünikasyon Mühendisliği}
{Lisans}
{İstanbul Teknik Üniversitesi}
	
\end{entrylist}

\section{meetuplar \& aktiviteler}

\begin{entrylist}

\entry
{2018}
{Perakende, Seyahat ve Üretimde Teknoloji Dönüşümü}
{}
{Microsoft Türkiye}

\entry
{2018}
{git \& Unit Test}
{}
{Doğuş Teknoloji}

\entry
{2017}
{Microsoft Yeni Teknolojiler Hackathonu}
{}
{Microsoft Türkiye}

\entry
{2017}
{XF RESTful Web Servis Tüketimi, Docker + .NET Core + Nginx İle Web Uygulamaları}
{}
{Microsoft Türkiye}

\entry
{2017}
{Tech Meetup: AI \& IoT Day}
{}
{Microsoft Türkiye}

\entry
{2017}
{GDG DevFest}
{}
{Kadir Has Üniversitesi}

\entry
{2017}
{Gerçek Blockchain Senaryoları}
{}
{Microsoft Türkiye}

\entry
{2016}
{Güvenli Kodlama}
{}
{Biznet Bilişim}

\entry
{2016}
{Agile Scrum}
{}
{agile42}

\end{entrylist}
\end{document}